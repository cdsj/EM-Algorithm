% Copyright 2004 by Till Tantau <tantau@users.sourceforge.net>.
%
% In principle, this file can be redistributed and/or modified under
% the terms of the GNU Public License, version 2.
%
% However, this file is supposed to be a template to be modified
% for your own needs. For this reason, if you use this file as a
% template and not specifically distribute it as part of a another
% package/program, I grant the extra permission to freely copy and
% modify this file as you see fit and even to delete this copyright
% notice. 

\documentclass[noindent]{beamer}
\usepackage{ctex}
\usepackage{amsmath,amsfonts,amssymb,amsthm}
\usepackage{clrscode3e}
% There are many different themes available for Beamer. A comprehensive
% list with examples is given here:
% http://deic.uab.es/~iblanes/beamer_gallery/index_by_theme.html
% You can uncomment the themes below if you would like to use a different
% one:
%\usetheme{AnnArbor}
%\usetheme{Antibes}
%\usetheme{Bergen}
%\usetheme{Berkeley}
%\usetheme{Berlin}
%\usetheme{Boadilla}
%\usetheme{boxes}
%\usetheme{CambridgeUS}
%\usetheme{Copenhagen}
%\usetheme{Darmstadt}
%\usetheme{default}
%\usetheme{Frankfurt}
%\usetheme{Goettingen}
%\usetheme{Hannover}
%\usetheme{Ilmenau}
%\usetheme{JuanLesPins}
%\usetheme{Luebeck}
\usetheme{Madrid}
%\usetheme{Malmoe}
%\usetheme{Marburg}
%\usetheme{Montpellier}
%\usetheme{PaloAlto}
%\usetheme{Pittsburgh}
%\usetheme{Rochester}
%\usetheme{Singapore}
%\usetheme{Szeged}
%\usetheme{Warsaw}

\title{EM Algorithm Tutorial}

% A subtitle is optional and this may be deleted
\subtitle{Expectation Maximization}

\author{李修成}
%\author{F.~Author\inst{1} \and S.~Another\inst{2}}

\institute[HIT]
{
    Department of Computer Science\\
    Harbin Institute of Technology
}

%\institute[Universities of Somewhere and Elsewhere] % (optional, but mostly needed)
%{
%  \inst{1}%
%  Department of Computer Science\\
%  University of Somewhere
%  \and
%  \inst{2}%
%  Department of Theoretical Philosophy\\
%  University of Elsewhere
%}

\date{}
%\date{Conference Name, 2013}

\subject{Theoretical Computer Science}

%\AtBeginSubsection[]
%{
%  \begin{frame}<beamer>{Outline}
 %   \tableofcontents[currentsection,currentsubsection]
%  \end{frame}
%}

% Let's get started

\begin{document}
\begin{frame}
  \titlepage
\end{frame}

\begin{frame}{Outline}
  \tableofcontents
\end{frame}

\section{EM算法的提出}
\begin{frame}{为什么要有EM算法}
\begin{itemize}
\item {
        EM算法解决的是含有隐含变量概率模型的参数估计问题(极大似然估计)
}

\item {
 考虑如下概率模型参数估计
\begin{equation}
\ell(\theta)=\log p(x;\theta) = \log  \sum_{z}p(x,z;\theta)
\end{equation}
问题难以用极大似然估计求解参数!难在$\log$与$p$之间有个$\Sigma$!
}
\pause
\item {
  Gaussian Discriminant Analysis model 高斯判别分析
}
\item {
  Gaussian Mixture Model 高斯混合模型
}
\end{itemize}
\end{frame}

\begin{frame}{GDA(高斯判别分析),MLE易解}
\begin{block}{高斯判别分析}
已知有多个正态分布,每个样本数据由一个正态分布产生,生成数据的正态分布已知。给出训练数据,现要估计参数
\end{block}
$p(x^{(i)},z^{(i)}) = p(x^{(i)}|z^{(i)})p(z^{(i)})$.
 $z^{(i)} \sim Multinomial(\phi) (where \ \phi_j \geq 0,
\sum_{j=1}^k \phi_j = 1$, and the parameter gives $p(z^{(i)}=j)$), and $(x^{(i)}|z^{(i)}=j) \sim N(\mu_j,\Sigma_j)$.
\end{frame}

\begin{frame}
GDA似然函数
\begin{equation}
\begin{split}
    \ell(\phi,\mu,\Sigma) &= \sum_{i=1}^{m}  \log p(x^{(i)},z^{(i)}; \mu , \Sigma ) \\
              &= \sum_{i=1}^{m} \log  p(x^{(i)}|z^{(i)}; \mu , \Sigma ) p(z^{(i)}; \phi ) \\
              &= \sum_{i=1}^{m} \log   \frac{1}{(2\pi)^{n/2}| \Sigma _j|^{1/2}}\exp(-\frac{1}{2}(x^{(i)}-\mu_j)^T\Sigma_j^{-1}(x^{(i)}-\mu_j))\phi_j 
\end{split}
\end{equation}
\end{frame}
\begin{frame}{MLE求解GDA}
Maximizing this with respect to $\phi, \mu, \Sigma$ gives the parameters:

\begin{itemize}
\item $\phi_j =  \frac{1}{m} \sum_{i=1}^m 1\{z^{(i)}=j\}$
\item $\mu_j = \frac{\sum_{i=1}^m 1\{z^{(i)}=j\} x^{(i)}}{\sum_{i=1}^m1\{z^{(i)}=j\}}$
\item $\Sigma_j = \frac{\sum_{i=1}^m 1\{z^{(i)}=j\}(x^{(i)}-\mu_j)(x^{(i)}-\mu_j)^T
}{\sum_{i=1}^m 1\{z^{(i)}=j\}}$
\end{itemize}
\end{frame}
\begin{frame}{GMM(高斯混合模型),MLE难解}
\begin{block}{高斯混合模型}
已知有多个正态分布,每个样本数据由一个正态分布产生,但具体是哪一个,我们无法观测(含有隐含变量)。给出训练数据,现要估计参数
\end{block}
$p(x^{(i)},z^{(i)}) = p(x^{(i)}|z^{(i)})p(z^{(i)})$.\\
 $z^{(i)} \sim Multinomial(\phi) (where \ \phi_j \geq 0,
\sum_{j=1}^k \phi_j = 1$, and the parameter gives $p(z^{(i)}=j)$), and $(x^{(i)}|z^{(i)}=j) \sim N(\mu_j,\Sigma_j)$.

\end{frame}

\begin{frame}
GMM似然函数
\begin{equation}
\begin{split}
   \ell(\phi,\mu,\Sigma)  &= \sum_{i=1}^{m}  \log p(x^{(i)},z^{(i)}; \mu , \Sigma ) \\
              &= \sum_{i=1}^{m} \log  \sum _{z^{(i)}=1}^{k} p(x^{(i)}|z^{(i)}; \mu , \Sigma ) p(z^{(i)}; \phi ) \\
              &= \sum_{i=1}^{m} \log  \sum _{j=1}^{k} \frac{1}{(2\pi)^{n/2}| \Sigma _j|^{1/2}} \exp(-\frac{1}{2}(x^{(i)}-\mu_j)^T\Sigma_j^{-1}(x^{(i)}-\mu_j))\phi_j 
\end{split}
\end{equation}
\end{frame}

\section{EM算法的推导}
\begin{frame}
\begin{itemize}
\item {
怎么办? \pause Jensen's Inequality
\begin{equation}
\begin{split}
\ell(\theta)=\log p(x;\theta) &=  \log  \sum_{z}p(x,z;\theta) \\
			   &= \log \sum_{z}Q(z)\frac{p(x,z;\theta)}{Q(z)} \\
                           &\geq \sum_{z}Q(z)\log \frac{p(x,z;\theta)}{Q(z)} 
\end{split}
\end{equation}
$Q(z)$为关于隐含变量$z$的任意分布
}
\end{itemize}
\end{frame}
\begin{frame}
\begin{itemize}
\item { 
令 $\displaystyle \mathcal{L}(x,z;\theta) = \sum_{z}Q(z) \log \frac{p(x,z;\theta)}{Q(z)} $得到
\begin{equation}
\begin{split}
\ell(\theta)-\mathcal{L}(x,z;\theta) &= \log p(x;\theta)-\sum_{z}Q(z)\log \frac{p(x,z;\theta)}{Q(z)} \\
                        &= \sum_{z} Q(z) \log p(x;\theta)-\sum_{z}Q(z)\log \frac{p(x,z;\theta)}{Q(z)} \\
			&= -\sum_{z}Q(z)\log\frac{p(z|x)}{Q(z)} = D(Q(z) \parallel p(z|x; \theta ))
\end{split}
\end{equation}
}
\item {
似曾相识!\pause LDA!
}
\item {
$\ell(\theta)-\mathcal{L}(x,z;\theta)$得到的是$Q(z)$和$p(z|x)$的KL距离,又称相对熵。
}
\item {
\begin{equation}
\ell(\theta) = \mathcal{L}(x,z;\theta) + D(Q(z) \parallel p(z|x; \theta ))
\end{equation}
}
\end{itemize}
\end{frame}

\begin{frame}
\begin{itemize}
\item {
由于$\mathcal{L}(x,z;\theta)$消除了$\ell(\theta)$中$\log$后面的$\sum$,因此它是容易计算的
}
\item{
由于$KL(q \parallel p) \geq 0$,故$\mathcal{L}(x,z;\theta)$为$\ell(\theta)$的下界
}
\item {
用$\mathcal{L}(x,z;\theta)$去逼近$\ell(\theta)$,而$\ell(\theta) \geq \mathcal{L}(x,z;\theta)$\\
故当$D(Q(z) \parallel p(z|x;\theta)) = 0$时,下界函数$\mathcal{L}(x,z;\theta)$可以最大化\\
\pause
EM算法的精髓!
}
\item {
显然,当$Q(z) = p(z|x;\theta)$时$\mathcal{L}(x,z;\theta)$极大化
}
\end{itemize}
\end{frame}

\section{EM Algorithm}
\begin{frame} {EM Algorithm}
\begin{codebox}
    \li Repeat until convergence \{
        \Do
    \li      \bf{E-step}: 
    \li			For each i, set
             \Do 
    \li                   $Q_{i}(z^{(i)}) := p(z^{(i)}|x^{(i)};\theta)$
             \End
    \li      \bf{M-step}:
    \li			 Set
             \Do

    \li			 $\theta := arg\ max_{\theta}  \sum_{i} \sum_{z}Q_{i}(z^{(i)}) \log \frac{p(x^{(i)},z^{(i)};\theta)}{Q_{i}(z^{(i)};\theta)}$
             \End
        \End
    \li \}
\end{codebox}
其中\bf{M-step}亦可为,$\theta := arg\ max_{\theta}  \sum_{i} \sum_{z}Q_{i}(z^{(i)}) \log p(x^{(i)},z^{(i)};\theta)$\\
因为,$\sum_{i} \sum_{z}Q_{i}(z^{(i)}) \log Q_i(z^{(i)})$是一个与$\theta$独立的常量
\end{frame}

\section{EM算法收敛性的证明}
\begin{frame} {EM算法收敛性的证明}
由于
\begin{equation}
\ell(\theta) \geq  \sum_{i}  \sum_{z}Q_{i}(z^{(i)}) \log \frac{p(x^{(i)},z^{(i)};\theta)}{Q_{i}(z^{(i)})} 
\end{equation}
$\Rightarrow$
\begin{equation}
\begin{split}
\ell(\theta^{(t+1)}) &\geq  \sum_{i}  \sum_{z}Q_{i}^{(t)}(z^{(i)}) \log \frac{p(x^{(i)},z^{(i)};\theta^{(t+1)})}{Q_{i}^{(t)}(z^{(i)})} \\
                  &\geq \sum_{i}  \sum_{z}Q_{i}^{(t)}(z^{(i)}) \log \frac{p(x^{(i)},z^{(i)};\theta^{(t)})}{Q_{i}^{(t)}(z^{(i)})} \\
                  &= \ell(\theta^{(t)})
\end{split}
\end{equation}
其中第2步成立是因为,$\theta^{t+1}$是由第$t$次迭代经极大似然估计得到。
\end{frame}

\section{EM算法的应用:求解GMM(高斯混合模型)}
\begin{frame}{EM算法应用:求解GMM(高斯混合模型)}
\alert{E-step:}令
\begin{equation}
\begin{split}
    w_{j}^{(i)} &= Q_{i}(z^{(i)}=j) \\
                &= p(z^{(i)}=j|x^{(i)}; \phi , \mu , \Sigma )\\ 
                &= \frac{p(x^{(i)}|z^{(i)}=j; \mu , \Sigma ) p(z^{(i)}=j; \phi )}{ \sum _{l=1}^{k}p(x^{(i)}|z^{(i)}=l; \mu , \Sigma ) p(z^{(i)}=l; \phi )}
\end{split}
\end{equation}
\end{frame}

\begin{frame}
\alert{M-step:}似然函数
\begin{equation}
\begin{split}
    \ell(\theta) &= \sum_{i=1}^{m}  \sum _{z}Q_{i}(z^{(i)}) \log p(x^{(i)},z^{(i)}; \mu , \Sigma ) \\
              &= \sum_{i=1}^{m}  \sum _{j=1}^{k}Q_{i}(z^{(i)}=j) \log p(x^{(i)}|z^{(i)}=j; \mu , \Sigma ) p(z^{(i)}=j; \phi ) \\
              &= \sum_{i=1}^{m}  \sum _{j=1}^{k}w^{(i)}_{j} \log \frac{1}{(2\pi)^{n/2}| \Sigma _j|^{1/2}} \exp(-\frac{1}{2}(x^{(i)}-\mu_j)^T\Sigma_j^{-1}(x^{(i)}-\mu_j))\phi_j 
\end{split}
\end{equation}
下面进行参数极大似然估计:
\begin{itemize}
\item $\mu_j$
\item $\phi_j$
\item $\Sigma_j$
\end{itemize}
\end{frame}

\begin{frame}{一些矩阵论中的知识}
%在进行参数估计这场战斗之前,先准备点克敌制胜的武器!
\begin{itemize}
\item $ \nabla_x x^T A x = 2Ax$, $A$ is symmetric
\item $ \nabla_x^2 x^TAx = 2A$, $A$ is symmetric
\item $ \nabla_x b^T x = b$
\item $ \nabla_A \log|A| = A^{-1}$, $A$ is invertible
\item $ \nabla_A x^T A x = x x^T$, $A$ is $n \times n$ matrix
\end{itemize}
\end{frame}

\begin{frame}{$\mu_j$的估计}
\begin{equation}
    \begin{split}
       \nabla _{\mu_j} \ell(\theta) &= \nabla _{\mu_j} \sum_{i=1}^m  \sum_{j=1}^k w_j^{(i)}\frac{1}{2}(x^{(i)}-\mu_j)^T  \Sigma _j^{-1}(x^{(i)}-\mu_j) \\
                                  &=  \sum_{i=1}^m w_j^{(i)} \Sigma_j^{-1}(x^{(i)}-\mu_j) = 0\\
                            \mu_j &= \frac{\sum_{i=1}^m w_j^{(i)}x^{(i)}}{\sum_{i=1}^m w_j^{(i)}}
    \end{split}
\end{equation}
\end{frame}

\begin{frame}{$\phi_j$的估计}
\begin{equation}
    \begin{split}
        \nabla _{ \phi _j} \ell(\theta) &= \nabla _{ \phi _j} \sum_{i=1}^m  \sum_{j=1}^k w_j^{(i)} \log \phi _j \\
                                    &= \sum_{i=1}^m \frac{w_j^{(i)}}{\phi_j}
    \end{split}
\end{equation}
\begin{itemize}
\item 解不下去了,梯度$\nabla _{ \phi _j} \ell(\theta)  > 0$,似乎只要使$\phi_j$取无穷大,似然函数就可以无限大!
\pause
\item 哪有那么好的事!别忘了还有约束条件$\sum_{j=1}^k \phi_j = 1$! 
\pause
\item 什么? \alert{约束条件}!求\alert{极大值}!无限遐想中$\ldots$\\
\pause
\item 想到了Lagrangian,因为目标函数是凸的嘛!\pause(其实是凹的,正好可以求极大值)
\end{itemize}
\end{frame}

\begin{frame}{$\phi_j$的估计}
构造Lagrangian函数
\begin{equation}
\mathcal{L}(\phi) = \sum_{i=1}^m \sum_{j=1}^k w_j^{(i)} \log \phi_j +\beta(\sum_{j=1}^k \phi_j - 1) 
\end{equation}
求偏导
\begin{equation}
    \begin{split} 
        \frac{\partial \mathcal{L}}{\partial \phi_j} &=  \sum_{i=1}^m \frac{w_j^{(i)}} {\phi_j} + \beta  \Rightarrow 
      -\beta \phi_j = \sum_{i=1}^m w_j^{(i)} \\
      -\beta \sum_{j=1}^k \phi_j &= \sum_{i=1}^m \sum_{j=1}^k w_j^{(i)} = m \Rightarrow -\beta = m \\
    \phi_j &= \sum_{i=1}^m \frac{w_j^{(i)}}{m}
    \end{split}
\end{equation}
\end{frame}

\begin{frame} {$\Sigma_j$的估计}
令$S_j = \Sigma_j^{-1}, b = (x^{(i)}-\mu_j)$ \\
\begin{equation}
    \begin{split}
    \nabla_{s_j} \ell(\theta) &= \nabla_{s_j} \sum_{i=1}^m w_j^{(i)}(\frac{1}{2}\log|S_j|-\frac{1}{2}b^T S_j b) \\
                          &= \frac{1}{2}\sum_{i=1}^m w_j^{(i)}(S_j^{-1}-bb^T) = 0\\
                 S_j^{-1} &= \frac{\sum_{i=1}^m w_j^{(i)} bb^T}{\sum_{i=1}^m w_j^{(i)}} \\
                 \Sigma_j &= \frac{\sum_{i=1}^m (x^{(i)}-\mu_j)(x^{(i)}-\mu_j)^T}{\sum_{i=1}^m w_j^{(i)}}
    \end{split}
\end{equation}

\end{frame}

\section{EM算法的应用:HMM(Hidden Markov Model)参数推断}
\begin{frame}{HMM(Hidden Markov Model参数推断}
\begin{itemize}
\item 用Stanford CS229的讲义
\end{itemize}
\end{frame}


\section{EM算法的推广:变分EM算法(Varitional EM)}
\begin{frame}{EM算法的推广:变分EM算法(Varitional EM)}
\begin{itemize}
\pause
\item 由于昨晚去喝酒了,而且喝得挺多,所以这部分内容没来的及做
\item 回忆$\mathcal{LDA}$(Latent Dirichlet Allocation)
\item $\ldots$
\end{itemize}
\end{frame}
\end{document}















